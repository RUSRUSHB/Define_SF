
\documentclass{ctexart}
\begin{document}

\title{定义科幻:关于科幻定义的论述合集——《科幻文学论纲》摘抄}
\maketitle
% \begin{abstract}
% 这是简介及摘要。
% \end{abstract}

\section{前言}
本人受群友讨论启发,依照《科幻文学论纲》等摘抄了一些与科幻的定义有关的论述。

本文内容是“与科幻的定义有关的论述”而不是“科幻的定义”。首先,科幻本身就是复杂的、“一刻也没有停止成长的”。其次,论述者可能根本就没有给科幻下定义的主观意愿:他们可能只是在描述自己观察到的现象,甚至可能只是在表达期望。“时代条件”“个人视角”等等老生常谈在此就不再赘述。除此之外,还有人认为科幻是不能被定义的。

本文不可能告诉你科幻的定义是什么,但是你读完之后可以得出自己的答案。

由于本人暂无精力阅读所有文段的原始文本,所作的摘抄可能与原文有些出入。

本文约7000字,欢迎收藏保存。
\tableofcontents
\section{《科幻文学论纲》摘抄}

本书作者:吴岩

\subsection{梁启超}
“哲理科学小说”是一种“专借小说以发明哲学及格致学”的读物。

\begin{itemize}
    \item 有深度哲理(深刻揭示宇宙和人生深层规律,把握自然与社会运演方向) 

    \item 有全新视野(具有独特的表现手法和叙事空间)

    \item 兼顾科学和哲理,并以高深学理为核心
\end{itemize}

\subsection{鲁迅}
科幻小说有“经以科学,纬以人情”的文本构造方式。

\subsection{海天独啸子}
科幻作品应该是“高尚之理想,科学之观察”的总和。

\subsection{小说林社}

科幻小说的特点是“启智秘钥,阐理玄灯”。



\subsection{碧荷馆主人}

(有两部未来小说)发达是借助了科学,科学是小说的材料。但科幻小说其实不是科学讲义,作者也不是科学专家。

\subsection{管达如}

普通小说的要领较虚,而科学的要领却是实征(证),二者似不相容。但由于科学发展已经创造了奇迹,所以,事奇斯文奇,这就是科幻之所在。

\subsection{成之}

科学小说是借小说输进科学智识,是趣味教育之作。

\subsection{清华小说研究社}

(威尔斯《世界大战》)在读者共同承认的前提下进行构造。

\subsection{瞿世英}

(威尔斯和左拉都是)以科学家的实验为观察方法创作作品。这种作品以知识为依据,能破除了空中楼阁的玄想。只有在科学势力所不及的地方,直觉和理想才能参与小说创作。

\subsection{费明君}

(阿•托尔斯泰的幻想小说)善于用历史的姿态描绘出过去、现在、未来的人类生活。

\subsection{王石安}

科幻是阅读了科学技术著作之后的补充读物。

\subsection{郑文光《谈谈科学幻想》}

“教科书叙述着有益事物,给我们知识,文艺作品使我们思考,科学幻想作品则教我们去想象未来。”

科幻文学属于严肃作品。

“不应当把幻想小说理解为未来的精确预言。在许多情况下,幻想是运用了科学文艺形式。”

“我们想在未来看到这个那个......”启发人去朝向科学的愿望,朝向理想,是科幻作品的最终目标。

\subsection{O.胡捷《论苏联科学幻想读物》}

必须创作出与资本主义生产方式具有显著差异的新生产社会,展现出这个社会中的人的关系,正确地预见科学创新,并给青少年普及科学知识。

\subsection{布•略普诺夫《技术最新成就与苏联科学幻想读物》}

\subsubsection{目的}
\begin{enumerate}
    \item 回答“这些未来工作到底会怎样?科学工作者对未来的想象是什么?”科幻作家与科学家期望同构。
    \item (一些科幻作品)用小说或特写的形式,为读者展开一幅化幻想为现实的图景。
    \item 科幻作家们“正在未经阐明的领域内进行它的研究工作”。
\end{enumerate}

\subsubsection{内容}
\begin{enumerate}
    \item 幻想作品中所描写的、已经实现的事物,在科技中不过刚刚有眉目,因此往往推动发明家去解决问题。有时,幻想家的大胆想象,远远超过时代的技术。
    \item “虽然说反过来:美丽的空想与科学很远,任何时候没有实现的可能。但科学技术思想的最新成就,却‘出乎意外地’成了这一类中某些作品的基础,结果,空中楼阁变成了有科学根据的幻想了。”
\end{enumerate}


\subsection{斯•波尔塔夫斯基《论科学幻想作品中一些悬而未决的问题》}

原有对科幻的定义是“描写出在写书的那段时间中不可能实现的事物”。然而,当前的问题是,幻想越来越难于赶上科学发展的速度。鉴于此,一些人提出未来预测方面的“取消论”,认为,既然科幻小说无法预测未来,就应该去普及科技成就。

波尔塔夫斯基反对取消论,他指出,科幻不单单是为预见而做,这种文学的内容应该是多方面的,特别是进入社会主义之后,技术虽然是生活中的必要环节,但提高小说的社会意义才是最重要的事情。纯粹写技术的科幻作品,就是脱离现实的作品。

别利亚耶夫提出“把不存在的东西描写成为已经存在的”,此乃科幻作品的任务和特征。但是,这样的定义会将社会未来小说或民间故事包容进去,不适合科幻定位。

苏联大百科全书认为科幻是“实际上还没有实现的科学发现和发明,但科学技术已有的发展一般已为它的实现准备了条件”。这样的定义无法涵盖那些描写回到过去的科幻作品。况且,波尔塔夫斯基认为,以发明作为基础的科幻作品已属过时。

科幻文学必须是人的文学。“在不久的将来,不是发明的本身,而是人在利用它的无限可能性之中的组织作用,将不可避免地成为创作科学幻想作品的基础。”

《阿爱里塔》预料了一种科幻创作的新路:那就是以人为主导,以技术作为从属。科幻作家应该根据马克思对社会发展的看法去研究如何描写未来的人。而这里所谓的未来之人,指的是在阶级消灭之后,人的空闲增加,而创造性活动成为了人的主要活动。

用“科学的可能性”去局限科幻小说的做法,是不可取的,是不了解科幻的性质、规律,不了解社会主义现实主义的无限可能性造成的。




\subsection{卡赞采夫}

    科幻创作是一种创新过程

    科学幻想跟科学的假设相近。它可以由假设产生,也可以产生假设。

    读者是苏联科幻作品中发明过程的参与者。在一种广义的科幻服务现实论的指引下,叙述被实现着的幻想,将要成为苏联科幻作品发展的主线。

    不应将科幻局限在现实,应该对现实抱着批评的态度。

    “苏联幻想作家的使命是:创作和我们时代相称的作品。在谈论明天的时候,不要落在今天的后面。幻想应该奔放而不受羁绊,语言应该精练,人物应该鲜明有力,能以其模范行为和思想去教导青年……”



\subsection{苏联科幻理论(的一种简要概括)}

    科幻文学应该是科学发现的先导,应该撰写社会主义和共产主义的新人。



\subsection{郑文光《往往走在科学发明的前面——谈谈科学幻想小说》}

    科幻是一种描写未来的文学式样,这种文学应该跟科学具有紧密的关系。“科学幻想小说就是描写人类在将来如何对自然作斗争的文学式样。”因为科学使幻想成为现实,因此,科学是科幻产生的基础。科幻要立足科学理论,且必须有科学根据。科学幻想小说作者经常利用科学家们的一些天才的、尚未付诸实践的思想和设计去撰写作品。

    虽然科幻必须有科学的基础,但作品并不一定要寻求精确的科学验证。“这绝不是说,科学幻想小说是未来人类的生产活动和生活的最精确的预言。因而,科学幻想小说的作者就无需像科学家那样依靠千百次观测、反复的实验、穷年累积的计算去建立科学的假说,只要不违反基本的科学原理,作家完全有权利在作品中加进自己的想象,自己的愿望,自己的天才臆测。想象力,这是一切文学作品中不可缺少的重要因素,在科学幻想小说中尤其如此。在这个意义上说,科学幻想小说正是继承了古典的神话和民间传说的传统,而成为具有充分浪漫主义特点的一个新的文学类型。”不寻求精确,也就意味着科幻允许在技术问题上违反科学(原理)。当然,不是说所有科幻小说都存在着科学上的问题或违背。一些科幻作家可以采取大胆假设来阐述卓越的科学思想。

    科幻的感染力来源于小说的故事、文字、形象和其中的精神力量。科幻不同于教科书和科学文艺读物,这类作品是通过文字感染力量和美丽动人的故事情节,形象地描绘现代科技无比的威力,指出人类光辉灿烂的远景。用美妙想象力启发和培养科学爱好,号召人们在征服自然中立功并向科学技术进军。



\subsection{叶永烈《论科学文艺》}

    科学幻想小说是通过小说来描述奇特的科学幻想,寄寓深刻的主题思想,具有“科学”、“幻想”、“小说”三要素,即它所描述的是幻想,而不是现实;这幻想是科学的,而不是胡思乱想;它通过小说来表现,具有小说的特点。


\subsection{刘后一}

    “科学之谜还多得很,甚至愈来愈多。很多问题连科学家都还在争论中,能要求一个科普作家什么都精通么?”



\subsection{高士其}

    科学小说或科幻小说,是以小说的体裁,描写人在科学领域内的实践活动的,它应具有小说的特点。它有故事情节,有典型形象的塑造,有人物性格的刻画。写人类探索、认识和征服自然活动使科幻小说具有丰富的内容。



\subsection{郑公盾}

    “科学文艺,是科学,也是文艺。”“科学文艺创作,首先是为特定的科学知识、科学内容服务的。科学文艺倘不能表现特定的科学主题,描写的是不科学、伪科学,甚至是反科学的东西,那当然谈不上是科学文艺作品。其次,科学文艺又必须具有一般文艺作品的特性,首先它要通过一系列艺术形象的描写来表现科学,使读者情不自禁地、潜移默化地受到感染和教育。”



\subsection{饶忠华}

    “两个构思”理论:科幻小说与普通小说不同,普通文学作品中只有一个人文构思,而科幻作品在这个人文构思之外,还有一个科学幻想的构思。任何科幻作品都必然有两个构思,而成功作品应该是两者结合的典范。



\subsection{王逢振《西方科学小说浅谈》}

    “一般来说,优秀的科学小说具备以下两点:首先符合当代的科学事实,其次在预示科学发展方面有突出的见解。科学小说与科学常常是一致的,而且许多科学发明没有应用之前,就在科学小说里得到描写。”

    “当代科学小说涉及到科学的各个方面。然而科学小说毕竟只是利用文学来表现科学技术的发展和它对社会的影响,并不是对科学定理做严密的论证。因此,它常常包含这样一些概念:试验的证据可以在其他时间或地点再现,试验的结果可以脱离试验而独立出来,理智和推理的结果可以表示决定性的预见,并且测量的参数和变数可以根据需要而加以改变。当然,这些概念在小说里常常彼此矛盾,但并不影响小说所要表现的主题。”

    “不少科学小说的作者本身就是科学家或科技工作者,他们的作品常常是科学研究和试验的真实记录。”恰恰是这种跟科学的无限接近,导致了西方科幻作品跟科学之间的那种深入和广泛的联系。



\subsection{《科幻文学论纲》23页18行}

    科幻文学是古典冒险文学和情节小说的继承者。



\subsection{中国20世纪70-80年代主要的科幻争论}

    略



\subsection{实用派批评理论}

\subsubsection{知识科普}

\begin{enumerate}
    \item 普及科学知识
        \begin{itemize}
            \item 科幻文学是一种科普文学,用于普及或传播科学知识,以达成公众理解科学目的。
            \item 鲁迅《月界旅行•辨言》:“盖胪陈科学,常人厌之,阅不终篇,辄欲睡去,强人所难,势必然矣。惟假小说之能力,被优孟之衣冠,则虽析理谭玄,亦能浸淫脑筋,不生厌倦。”
            \item 顾均正《在北极底下》:“那么我们能不能,并且要不要利用这一类小说来多装一点科学的东西,以作普及科学教育的一助呢?我想这工作是可能的,而且是值得尝试的。”
            \item O.胡捷:“(科幻)是用文艺体裁写成的——它用艺术性的、形象化的形式传播科学知识。”
            \item 李赫兼斯坦:“科学幻想读物是普及科学知识的一种工具。”
        \end{itemize}
    \item 普及科学思想方式、认知方法、世界观、精神状态等
        \begin{itemize}
            \item 童恩正《谈谈我对科学文艺的认识》:(包含着科幻小说的)“科学文艺”的目标,是“普及科学的人生观”。
        \end{itemize}
\end{enumerate}

\subsubsection{未来愿望}

\begin{enumerate}
    \item 人对未来状况的某种期待、未来学的变体
        \begin{itemize}
            \item 20世纪20-50年代的苏联批评界:科幻文学是描写或普及假定的科学发现或技术发明
            \begin{itemize}
                \item 略普诺夫(见前文)
            \end{itemize}
            \item 美国(基于20世纪20-30年代的科幻创作实践以及凡尔纳科幻传统)
            \begin{itemize}
                \item 根斯巴克:科幻是“物质发展领域内的预言文学”
                \item 利文斯通:科幻是“未来学的重要组成部分”
            \end{itemize}
            \item 正在追赶现代化步伐的发展中国家
            \begin{itemize}
                \item 郑文光《往往走在科学的前面》:科幻小说应该可以引发新的科学研究
            \end{itemize}
        \end{itemize}
\end{enumerate}


\subsection{客观派批评理论}

\subsubsection{宏观}
\begin{enumerate}
    \item 达科·苏恩文
    \begin{itemize}
        \item 在其文学分类方法中,认知的/非认知的、自然主义的/陌生化的构成两个对子。科幻小说(和田园牧歌文学)是认知的(符合事物发展基本规律的)、陌生化的(描写人类独特想象的)。
        \item 科幻小说中的认知性并非永恒地存在,常常地,人们会采取认知逃避的态度对待现实。这样,科幻小说就成为矛盾冲突的现场,它在认知与逃避之间不断波动。这是一种“发达的矛盾修饰法”。
        \item 科幻文学是由陌生化和认知宰制/霸权(占据统治地位的文本构造方式也是小说的内容方式)的一种文学。恰恰是这种陌生化,使科幻文学产生了许多与现实不同的想象,但这些想象可以被认知过程所解释。科幻小说还具有很强的历史性。
    \end{itemize}

    \item 罗伯特•斯科尔斯
    \begin{itemize}
        \item 虚构世界对现实只有寓言性,但科幻的虚构跟奇幻或童话式的虚构有着巨大差别。因为它仍然跟现实世界之间存在着一种时间或空间上的联系,可能被一组现有的或参照现有科学规律的设定所限制,也可能被未来或异地的现实所参照。换言之,科幻小说不能随意安排故事,不能像撰写天堂、地狱、伊甸园、童话世界、大人国和小人国等式样的文学那样天马行空,因此,这样的寓言不是随意的自然性寓言,而是结构化的寓言。也就是说,科幻文学中的内容可以随意变换,但其形式一定是具有结构性的,而这种结构与现实主义文学之间又有着差异,不会严格受到当前的时间、空间、历史经验的左右。有关形式内容的理论,除了可以从宏观的方面进行构筑,还可以从主体性和文字语句等微观的层面进行构筑。
    \end{itemize}
\end{enumerate}

\subsubsection{微观}

\begin{enumerate}
    \item 赛缪尔•迪兰尼
    \begin{itemize}
        \item 从主体性和语言学角度来看,科幻文学之所以是科幻文学,主要是采用了具有现实性、推测性或模仿科学的话语。这样,“科学”置换了普通文学作品中针对事件进行指示性/表征的那些仅有隐喻的、甚至是根本无含义的话语语句。
        \item 通过其更广泛的领域修辞和语句组织,科幻文学的语句拓展了简单的报告性的语句,给事件的可能性增加了广度。
        \item 科幻小说的主体性的清晰程度,与自然主义小说、奇幻小说、报告文学都不相同。
    \end{itemize}
\end{enumerate}

\subsection{表现派批评理论}



    科幻作品表达了作家对未来的想象,这种想象聚焦于科学怎样影响社会方面
    \begin{itemize}
        \item 小约翰·W·坎贝尔:科幻是以故事形式,描绘科学应用于机器和人类社会时产生的奇迹。科幻小说必须符合逻辑地反映科学新发明如何起作用,究竟能起多大作用和怎样的作用。
        \item R. 布雷特纳:科幻小说是科学以及由此而产生的技术对人类影响所做的理性推断为基础的小说。
        \item 阿西莫夫:科幻小说是文学的一个分支,主要描绘虚构的社会,这个社会与现实社会的不同之处在于科技的发展性质和程度。科幻可以界定为处理人类回应科技发展的一个文学流派。
        \item 克拉克:“变化的文学”
        \item 詹姆斯·冈恩:科幻是文学的新品种,它描绘真实世界的变化对人们所产生的影响。它可以把故事设想在过去、未来或者某些遥远的空间,它关心的往往是科学或者技术的变化。它设计的通常是比个人或者小团体更为重要的主题:文明或种族所面临的危险。
        \item 罗伯特·海因莱因:在科幻小说中,作者表现了对被视为科学方法的人类活动之本质和重要性的理解。同时,对人类通过科学活动收集到的大量知识表现了同样的理解,并将科学事实、科学方法对人类的影响及将来可能产生的影响反映在他的小说中。
        \item 威廉·拉普询问的一些英语文学专业教授:科幻小说是“试图去预测未来技术进步对社会影响的一类故事”。而这类故事如果写好了,还将提供给读者一个与当前世界有所差异的“替代的世界”。
    \end{itemize}

\subsubsection{补充:思想实验理论}
\begin{itemize}
    \item 阿西莫夫:“纸上的社会实验”
    \item 海因莱因:“推断文学”
    \item 叶菲烈莫夫:“逻辑思考的文学”
    \item 勒古恩:科幻是“假如……则会……(what if)”。当你作出数不清的各种假如,世界的种种可能都可以在其中进行测试。幻想是一种带有虚拟性的概念,是一个针对现实的思想实验。
\end{itemize}


\section{《当我们谈论科幻时,我们在谈论什么?》摘抄}

本文章来源:https://zhuanlan.zhihu.com/p/47347491?utm\_id=0,来自科幻作家双翅目与机核电台主持人四十二、西蒙的播客对话:https://www.g-cores.com/radios/102374



\subsection{雨果·根斯巴克}

    科幻是凡尔纳、威尔斯和爱伦坡。



\subsection{约翰·坎贝尔}

    科幻小说包含了对技术社会的希望、梦想与恐惧。科幻是写在纸上的梦。



\subsection{R.海因莱因}

    科幻是speculative fiction(推断性文学)。推断是对未来事件的一个现实主义的推想或是演绎。



\subsection{冈恩}

    科幻小说就是社会小说。

\section{《我的科幻世界观》摘抄}

文章来源:豆瓣文章《我的科幻世界观——我认为科幻该如何定义与分类》
% TODO:
% https://www.douban.com/note/703215798/

    科幻是为抽象概念推演出的事物与冲突赋予先进科技的表象来推动剧情的故事。

    科学是以怀疑为基础的研究方法,但是科幻里的“科学”追求的却是故事内的自洽,两者在“信”的方向上可以说是完全相反。

\section{词典摘抄}



\subsection{《辞海》}

    依据科学技术上的新发现、新成就以及在这些基础上可能达到的预见,用幻想的方式描述人类利用这些新成果完成某些契机的新型小说



\subsection{《简明不列颠百科全书》}

    20世纪发展起来的一种文学体裁,这种体裁的小说以真实或想象的科学理论的发现为基础

\subsection{《苏联大百科全书》}

实际上还没有实现的科学发现和发明,但科学技术已有的发展一般已为它的实现准备了条件。

\section{《科幻文学论纲》作者观点摘抄}

\subsection{吴岩}
    “……如果说科幻文学是人类表达另类现代性蓝图的某种谋划……”

    “(在运用权力分析方法和作家簇分析方法之后,)在我的世界里,科幻文学的发展不再简单地成为一段人类如何探索自然、战胜自身、走向宇宙、面向未来的浪漫历史,它变成了一段现代化进程中关涉权力的斗争历程,从一个侧面展现了以科学为主要特征的当代社会中低位者的迷惘、痛苦、挣扎和反抗。”

    “科幻小说其实是科技变革的时代里,受到各类社会压制的边缘人通过作品对社会主流思想、主流文化和主流文学所进行的权力解构,而这种解构的方式,就是欲望上的对抗化、内容上的陌生化、形式上的方法化以及人物的种族化。”

    “科幻包含行动”

    “科幻文学是非主流人群采用非主流的方式在现代化过程中发出的喊声,由于确实处于文学和社会生活的非主流位置,因此,对无论从思想性、情感性、行动性和文本构造方式上,科幻文学都具有独特的特征。科幻作家在作品中呈现的诸多抱怨、反抗、建构、反思,通过认知系统内的实验去面对社会的方式,抚慰了时代变迁下受伤的心灵,为未来的社会发展带去了有价值的思考和体验。而科幻所创造的想象的社会产品、科幻所营造的种种想象的图景,给人类以鼓励和警示。”

    注:此部分未完待续。

\section{补充}

    因笔者能力和精力有限,本文难免存在纰漏。如有发现错误,请务必指出。如有其他方面的建议(包括格式),也欢迎提出。

    在阅读《科幻文学论纲》前/后,你还可以收听【科幻文学的过去、现在与未来:与吴岩老师聊《科幻文学论纲》 BV1Mq4y1E7Bi】。

    本文首发于微信公众号【实践与检索】
    % TODO:

\end{document}